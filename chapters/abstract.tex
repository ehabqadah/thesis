
\par In many application domains, such as maritime surveillance, financial services, network monitoring, and sensor networks, massive amounts of streaming data are being generated in real-time. The records of these streams can be encoded as events.  However, in order to benefit from the live streaming events, there is a need for systems that enable the real-time stream processing and analytics tasks at large-scale. For instance, predicting full matches of complex patterns from the massive streaming events is an important utility for the decision making process. Such a utility allows to react proactively to the new situations and to improve the effectiveness of the operational tasks.

\par In this thesis, we present the design, implementation, and evaluation of a scalable prediction system for user-defined patterns over multiple massive streams of events. The proposed system is based on a novel approach of combining probabilistic event pattern prediction models on multiple predictor nodes with a distributed online learning protocol to continuously learn the parameters of a global prediction model in a communication-efficient way, and to share it among the predictors. For  scalability, the system is implemented on top of Apache Flink, a popular engine for distributed and large-scale stream processing.


% The patterns are defined in the form of regular expressions over the event types in the stream. The underlying model provides online predictions about when a pattern is expected to be completed within each event stream. 

\par  The key idea of our system is to enable the cooperative learning and information exchange between the distributed predictors by sharing a global prediction model, where the learning convergence is accelerated with less data for each predictor. We describe the distributed architecture and implementation of the proposed system along with the theoretical analysis that focuses on giving a probabilistic learning guarantee for the proposed synchronized global model. Also, we provide experimental results on synthetic event stream and real-world event streams in the context of maritime domain that prove the effectiveness of our proposed approach.


%outperformed a standard algorithm that works on data aggregated at a single location

%Predicting full matches of complex patterns from various real-time event streams is an important utility for the decision maker in many application domains such as maritime surveillance, financial services, and sensor networks. Such a utility allows him/her to proactively react to the new situations and enhances the operational decision making. An event stream is an unbounded collection of timely ordered data observations in the form of an attribute tuple that is composed of a value from finite event types along with other categorical and numerical features. For example, in the context of maritime surveillance, patterns prediction over real-time tracking streams of moving vessels is useful to alert maritime operation mangers about suspicious activities (e.g., fast sailing vessels near ports) before they happen, in this scenario, the event stream of a moving vessel consists of spatial-temporal and kinematic information along with the vessel's identification and its trajectory related event types. However, processing real-time streaming data is challenging since data streams are large in nature and continuously keep on coming at a high rate. 
%% we describe the design and implementation of a system called% 
%\par To this end, in this thesis, we present an online, distributed and scalable patterns prediction system over massive input event streams. The proposed approach is based on a novel approach that combines the  distributed online prediction protocol \citep{kamp2014communication} with the event forecasting with Pattern Markov Chain system\citep{alevizos2017event}, in order to provide an online and large-scale patterns prediction system over distributed real-time event streams. 
%
%\par We leverage an existing online probabilistic forecasting model, which consists of the event forecasting with Pattern Markov Chain~\citep{alevizos2017event}. This system gives the ability to predicate when a pattern within a stream of events will be fully matched. In addition, we integrate the distributed online synchronization protocol \citep{kamp2014communication} to provide distributed online learning capabilities between the prediction models in a communication-efficient manner. This protocol enables us to dynamically synchronize the distributed local prediction models of the multiple input event streams.
%
%In some practical applications the input event streams may belong to different distributions, we propose to divide the input event streams into similar groups, in order to combine the corresponding predictions models to construct a representative global model in each group. The aggregation operation refers to the synchronization operation (e.g., joint average of the local models) in the distributed online learning protocol, which is performed by a central coordinator that constructs and distributes a global prediction model for the input event streams based on the  local models. In addition, we are using one of the modern Big Data frameworks for stream processing i.e., Apache Flink \footnote{\url{https://flink.apache.org/}} to implement the proposed system.
%\par We aim to provide an architecture and implementation of a system for large-scale patterns prediction over multiple input event streams. Our approach exploits the collaborative learning from multiple input streams by combing their associated predication models in a dynamic way. 
%
%\par We will analysis and develop the proposed approach, and evaluate the effectiveness through empirical experiments over synthetic event streams and real-word data streams of moving objects, in particular, events streams related to trajectories of moving vessels, which are provided in the context of the datAcron project\footnote{\url{http://www.datacron-project.eu/}}.
%
%%Furthermore, we will try to provide probabilistic guarantees for the new proposed %method by performing theoretical analysis procedures. 


%points:
%1)We empirically demonstrate that our method outperforms conventional methods.
%2) Experimental results on synthetic and real-world event streams show the effectiveness of our proposed approach.
%3) we propose the integrated usage
