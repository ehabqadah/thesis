\chapter{Discussion}
\label{chapter:discussion}
In this chapter, we discuss the results of our system, and some of the aspects of underlying method. Also we give some proposals for future work.
\section{Result}
\section{Approach}
\section{Future Work}
\begin{itemize}[noitemsep]
	\item In some practical applications the input event streams may belong to different distributions, we propose to divide the input event streams into similar groups, in order to combine the corresponding predictions models to construct a representative global model in each group. The aggregation operation refers to the synchronization operation (e.g., joint average of the local models) in the distributed online learning protocol, which is performed by a central coordinator that constructs and distributes a global prediction model for the input event streams based on the  local models.
	\item temporal patterns 
	\item another communication media  
	\item theoretical analysis of dynamic protocol 
	\item another transition probabilities learning technique 
	\item another weighted sync operator
	
	
%	 it seems that predicted patterns currently concern individual objects (hence each one is monitored by a distinct predictor), although in reality there may be interactions and correlations among moving objects. As a future research direction, perhaps it would be worth briefly discussing whether and how this method could be extended to handle such correlated events involving multiple objects (e.g., a group of vessels heading to a place).
	
	
\end{itemize}