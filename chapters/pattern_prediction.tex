

\subsection{Pattern Prediction over Event Streams}

\par Several approaches have been proposed to formalize the task of event patterns (complex events) prediction over time-evolving data streams.  One common way to formalize this task is to assume that the stream is a time-series of numerical values, and the goal is to predict at each time point $t$ the next observations at some future points $t+1$, $t+2$, etc., (or even the output of some function of future values)~\cite{montgomery_introduction_2015}. 


%The task of forecasting over time-evolving streams of data can be formulated in various ways and with varying assumptions.
%One common way to formalize this task is to assume that the stream is a time-series of numerical values, and the goal is to forecast at each time point $n$ the values at some future points $n+1$, $n+2$, etc., (or even the output of some function of future values). 
% This is the task of time-series forecasting ~\citet{montgomery_introduction_2015}.

\par Another way to formalize the prediction task is to view streams as sequences of events,
i.e., tuples of multiple attributes, such as \textit{id}, \textit{event type}, \textit{timestamp}, etc., and the goal is to predict future events or  patterns of events. In this work, we focus on the latter definition of forecasting (prediction of complex event patterns).  

\par A relevant work of the detection of event patterns task has been established in the field of temporal pattern mining. Where events are defined as 2-tuples of the form \((\mathit{EventType}, \mathit{Timestamp})\). The goal is to extract patterns of events in the form of association rules \cite{agrawal_mining_1993} or frequent episode rules \cite{mannila_discovery_1997}. 

\par These rule-based methods have been extended in order to be able to learn rules for predicting event patterns. For instance, in \cite{vilalta_predicting_2002}, an association rule mining technique is introduced. This technique works as following. Firstly, it extracts sets of event types that frequently lead to a target event (i.e., a rare event) within a time window of a fixed size. Then, it uses them to build a rule-based prediction model. Also,  ~\citet{weiss1998learning} proposed another rule-based method to predict rare events in a stream, using a genetic algorithm to find all predictive patterns to form prediction rules.  


\par On the other hand, ~\citet{laxman_stream_2008} have proposed a probabilistic model for calculating the probability of the immediately next event in the stream. Which is achieved by combining each frequent episode \cite{mannila1997discovery} that presents a partially-ordered set of event types with a Hidden Markov Model (HMM). In addition, ~\citet{fahed_efficient_2014} have proposed episode rules mining algorithm which predicts distant events by generating a set of episode rules in the form \(P \rightarrow Q \) where $P$ and $Q$ are two episodes. These rules have a minimal antecedent (in number of events) and temporally distant consequent.

\par In \cite{zhou_pattern_2015} a mining method is presented that finds the frequent sequential patterns in a stream of events, which are then used to generate prediction rules. The event stream is processed into batches, in each batch the events are consumed to find the prefix matches from the discovered frequent patterns, to predict future events using different strategies of prediction scoring.

% add more details about the cep
\par Event pattern prediction has also attracted some attention from the field of complex event processing (CEP), where the CEP system consumes a stream of low-level events to detect patterns of events (composite events) that are defined using pattern-based languages, these languages provide logic, sequence, and iteration operators such as SQL-like languages \cite{Cugola:2012:PFI:2187671.2187677}. 
\par One such early approach is presented in \cite{muthusamy_predictive_2010}, which is based on converting the complex event patterns to automata, and subsequently, Markov chains are used in order to estimate when a pattern is expected to be fully matched. 
\par Moreover, ~\citet{alevizos2017event} have recently presented a similar approach, where automata and Markov chains are employed in order to provide future time intervals during which a full match of the pattern is expected with a probability above a confidence threshold. In this thesis, we leverage this method as the base prediction model for each input event stream (see Section ~\ref{sec:Event-Forecasting-PMC}). 