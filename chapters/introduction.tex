\chapter{INTRODUCTION}


\section{Motivation}
\par In recent years, technological advances have led to a growing availability of massive amounts of continuous streaming data (i.e., data streams observing events) in many application domains such as social networks \cite{reuter2012event,mathioudakis2010twittermonitor}, Internet of Things (IoT) \cite{miorandi2012internet}, user activities on the web \cite{banerjee2001clickstream,metwally2005duplicate} and maritime monitoring \cite{patroumpas2015event,laxhammar2010conformal}. These data streams provide the opportunity to implement reactive components within these domains.  For instance, the ability to detect and predict the full matches of a pattern of interest (e.g., a certain sequence of events), defined by a domain expert, is typically important for operational decision making tasks in the respective domains.

\par An event stream is an unbounded collection of time-ordered data observations in the form of a tuple of attributes that is composed of a value from finite event types along with other categorical and numerical attributes \cite{agrawal2008efficient,schultz2009distributed,zhou_pattern_2015}. As an illustrative example, consider the movement event streams in the context of maritime surveillance, the event stream of a moving vessel consists of spatio-temporal and kinematic information along with the vessel's identification and its trajectory related events, based on the automatic identification system (AIS) \cite{ais} messages that are continuously sent by the vessel. Therefore, leveraging event patterns prediction over real-time streams of moving vessels is useful to alert maritime operation managers about suspicious activities (e.g., fast sailing vessels near ports, or illegal fishing) before they happen. 

\par However, processing the real-time streaming data poses new challenges, since the data streams are large and distributed in nature and continuously arrive at a high rate \cite{Babcock2002,Flouris2017}. To deal with these data streams in a fast and efficient manner, a distributed stream processing framework \cite{Spark,Flink,Storm} is usually used to implement the streaming processing and analytic applications. 


\section{Thesis Overview}
% we describe the design and implementation of a system called% 
\par In this thesis, we present the design, implementation, and evaluation of a scalable and distributed system that provides obline pattern prediction over multiple real-time streams of events. The proposed approach is based on a novel method that combines a distributed online learning protocol \cite{dekel2012optimal,kamp2014communication} with an event forecasting method based on Markov chains \cite{alevizos2017event}. Our system introduces a new synchronization operator withing the distributed online learning protocol, which enables us to synchronize the the parameters of distributed pattern prediction models, where we first provide the theoretical aspects of this operation.

Furthermore, We implemented our system on top of the Big Data framework for stream processing Apache Flink \cite{Flink}, and the distributed streaming platform Apache Kafka \cite{Kafka}. We evaluate our proposed system over synthetic event streams, and real-world data streams of moving vessels, which are provided in the context of the datAcron project\footnote{\url{http://www.datacron-project.eu/}}.

In summary, the main contributions of this thesis are the following:

\begin{itemize}
	\item An architecture design of a distributed system for  event patterns prediction over massive event streams, alongside the implementation details on the top of Apach Flink and Apache Kafka.  
	\item We introduce a new model synchronization operation within a distributed online learning protocol, additionally,  we provide a theoretical analysis of the proposed synchronization operation, in which we derive an  efficiency probabilistic guarantee. 
	\item Experimental evaluation of the performance for the proposed methods in real- world  event streams and synthetic event streams.
  
\end{itemize}


\section{Publication}

Parts of this thesis have been published in \cite{Qadah}:\\ \\
\bibentry{Qadah}.

\section{Outline }

\par The rest of this thesis is structured as follows. Chapter ~\ref{chap:realred_work} provides a review  of the related work and used frameworks. In Chapter ~\ref{chapter:system}, we describe the problem of events pattern prediction, and our approach along with its practical and theoretical aspects.
 Chapter ~\ref{chapter:overview} presents how the proposed system was built, its architecture,  and the implementation details in Flink and Kafka.
 Chapter ~\ref{chapter:evaluation} presents the experimental setup and results of our system over event streams of moving vessels and synthetic event streams. We further
  discuss the results of our method and some potential future directions in Chapter ~\ref{chapter:discussion}. Finally, Chapter ~\ref{chap:conclusions} gives the overall conclusions of this thesis. 
	



%We discuss the related work and used frameworks in Section ~\ref{sec:realred_work}. In Section ~\ref{sec:system}, we describe the problem of pattern prediction, our proposed approach, and the architecture of our system. The implementation details on top of Flink are presented in Section ~\ref{sec:impl} and the experimental results in Section ~\ref{sec:results}. We conclude in Section ~\ref{sec:concl}.



%List of paper to check: 
%1) Distributed Complex Event Processing with Query Rewriting (intro,def)
