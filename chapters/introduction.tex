\chapter{Introduction}


\section{Motivation}
\par In the past decade, technological advances have led to a growing availability of massive amounts of continuous streaming data (i.e., data streams observing events) in many application domains such as social networks \cite{reuter2012event,mathioudakis2010twittermonitor}, Internet of Things (IoT) \cite{miorandi2012internet}, user activities on the web \cite{banerjee2001clickstream,metwally2005duplicate} and maritime monitoring \cite{patroumpas2015event,laxhammar2010conformal}. Where these data streams  used to be stored and processed later, while monitoring and processing them in real-time increases their valuable \cite{carney2002monitoring}.

 Moreover, these event streams provide the opportunity to implement reactive components within these domains.  For instance, the ability to detect and predict the full matches of a pattern of interest (e.g., a certain sequence of events), defined by a domain expert, is typically important for operational decision-making tasks in the respective domains.

\par An event stream is an unbounded collection of time-ordered data observations in the form of a tuple of attributes that is composed of a value from finite event types along with other categorical and numerical attributes \cite{agrawal2008efficient,schultz2009distributed,zhou_pattern_2015,flouris2017issues}.  On the other hand, event patterns are defined using a pattern language such as SQL-based \cite{schultz2009distributed} or regular expressions based \cite{alevizos2017event}.

\par As an illustrative example, consider the movement event streams of group of vessels in the context of maritime surveillance, the event stream of a moving vessel consists of spatio-temporal and kinematic information along with the vessel's identification and its trajectory related events, based on the Automatic Identification System (AIS) \cite{ais} messages that are continuously sent by the vessel. Therefore, leveraging event patterns prediction over real-time streams of moving vessels is useful to alert maritime operation managers about suspicious activities (e.g., fast sailing vessels near ports, or illegal fishing) before they happen \cite{patroumpas2015event}. 

\par However, processing the real-time streaming data poses new challenges, since the data streams are large and distributed in nature and continuously arrive at a high rate \cite{Babcock2002,Flouris2017}. To deal with these data streams in a fast and efficient manner, a distributed stream processing framework \cite{Spark,Flink,Storm} is usually used to implement the stream processing and analytic applications. 


\section{Thesis Overview}
% we describe the design and implementation of a system called% 
\par In this thesis, we present the design, implementation, and evaluation of a scalable and distributed system that provides online pattern prediction over multiple real-time streams of events. The proposed approach is based on a novel method that combines a distributed online learning protocol \cite{kamp2014communication}, with online probabilistic prediction models based on pattern Markov chain technique \cite{alevizos2017event} that forecast when an event pattern is expected to be matched in an event stream, where the patterns are defined in the form of regular expressions . 

\par In particular, we consider the setting when there are \emph{$K$} input event streams, and for each of these streams a local prediction model is built on a distinct node. As a result, \emph{$K$} models that are maintained separately on \emph{$K$} distributed predictor nodes each of which is consuming a single event stream.


 \par We propose to make the predictors of the event streams to collectively learn a global model in a distributed and communication-efficient fashion. Toward this, we adapt the distributed online learning protocol \cite{kamp2014communication}, where the pattern predictors \cite{alevizos2017event} send their local models to a coordinator node. We introduce a new synchronization operator that combines these independent local models to create a stronger global model and share it among the predictors.
  

\par In addition, we implemented our system on top of the Big Data framework for distributed stream processing Apache Flink \cite{Flink}, and the streaming platform Apache Kafka \cite{Kafka}. Furthermore, we evaluate our proposed system over synthetic event streams, and large-scale real-world data streams related to trajectories of moving vessels, which are provided in the context of the datAcron project\footnote{\url{http://www.datacron-project.eu/}}.\\

In summary, the main contributions of this thesis are the following:

\begin{itemize}
	
	\item We propose a new method of combining a distributed online learning protocol with pattern Markov chain predictors over multiple input event streams. It is based on enabling the collaborative learning and information exchange among the predictors.  
	
	\item We study the characteristics of the proposed method, by providing a theoretical analysis of the proposed synchronization operation within the distributed online learning protocol, in which we provide a probabilistic guarantee of the learning efficiency. 
	
	\item We describe the architecture of our distributed system for event patterns prediction over massive event streams, along with the building blocks for workflow processing. We also present the implementation details of  the system on the top of Apache Flink and Apache Kafka. 
	%The implementation is open
	%source9 source code: http:.....
	
	\item We conduct an extensive empirical evaluation of the proposed approach over real-world and synthetic streams of events.
  
\end{itemize}


\section{Publication}

Parts of this thesis have been published in \cite{Qadah}:\\ \\
\bibentry{Qadah}.

\section{Outline }

%\par The rest of this thesis is structured as follows. Chapter  provides a review  of the related work and used frameworks. In Chapter ~\ref{chapter:system}, we describe the problem of events pattern prediction, and our approach along with its practical and theoretical aspects.
% Chapter ~\ref{chapter:overview} presents how the proposed system was built, its architecture,  and the implementation details in Flink and Kafka.
% Chapter ~\ref{chapter:evaluation} presents the experimental setup and results of our system over event streams of moving vessels and synthetic event streams. Finally, we give the overall conclusions of this thesis, and we discuss some potential future directions in Chapter ~\ref{chap:conclusions}.
	
\par The rest of this thesis is organized as follows:\\
\\
\textbf{Chapter~\ref{chap:realred_work}} provides an overview of the related work on pattern prediction over event streams, distributed online learning techniques, and used Big Data frameworks.
\\
\\
\textbf{Chapter~\ref{chapter:system}} describes the problem of  pattern prediction over a single event stream and introduces the base used model. Then it presents the our method for pattern prediction over multiple input event streams by leveraging the distributed online learning protocol. Also, it provides the theoretical analysis and probabilistic learning guarantee of the proposed approach.  
\\
\\
\textbf{Chapter~\ref{chapter:overview}}  presents the architecture of the proposed system, and the implementation details in Flink and Kafka.
\\
\\
In \textbf{Chapter~\ref{chapter:evaluation}}, experiments and results over real-world event streams of moving vessels and synthetic event streams are presented.
\\
\\
Finally, conclusions are given in \textbf{Chapter~\ref{chap:conclusions}} along with potential directions for future work.



%We discuss the related work and used frameworks in Section ~\ref{sec:realred_work}. In Section ~\ref{sec:system}, we describe the problem of pattern prediction, our proposed approach, and the architecture of our system. The implementation details on top of Flink are presented in Section ~\ref{sec:impl} and the experimental results in Section ~\ref{sec:results}. We conclude in Section ~\ref{sec:concl}.



%List of paper to check: 
%1) Distributed Complex Event Processing with Query Rewriting (intro,def)
