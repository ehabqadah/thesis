\documentclass[]{article}

%opening
\title{extra abstracts}
\author{}

\begin{document}

\section{abstract1}

Event-based systems are rapidly gaining importance in many application domains ranging from real time monitoring systems in production, logistics and networking to complex event processing in finance and security. The event based paradigm has gathered momentum as witnessed by current efforts in areas including publish/subscribe systems, event-driven architectures, complex event processing, business process management and modelling, Grid computing, Web services notifications, information dissemination, event stream processing, and message-oriented middleware. The various communities dealing with event based systems have made progress in different aspects of the problem. The DEBS conference attempts to bring together researchers and practitioners active in the various sub communities to share their views and reach a common understanding.

\section{abstract1}
Today we are witnessing a dramatic shift toward a data-driven economy, where the ability to efficiently and timely analyze huge amounts of data marks the difference between industrial success stories and catastrophic failures. In this scenario Storm, an open source distributed realtime computation system, represents a disruptive technology that is quickly gaining the favor of big players like Twitter and Groupon. A Storm application is modeled as a topology, i.e. a graph where nodes are operators and edges represent data flows among such operators. A key aspect in tuning Storm performance lies in the strategy used to deploy a topology, i.e. how Storm schedules the execution of each topology component on the available computing infrastructure.

In this paper we propose two advanced generic schedulers for Storm that provide improved performance for a wide range of application topologies. The first scheduler works offline by analyzing the topology structure and adapting the deployment to it; the second scheduler enhance the previous approach by continuously monitoring system performance and rescheduling the deployment at run-time to improve overall performance. 

%very good%
Experimental results show that these algorithms can produce schedules that achieve significantly better performances compared to those produced by Storm's default scheduler.

\end{document}


