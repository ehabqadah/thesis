
\chapter{RELATED WORK AND BACKGROUND}
\label{sec:realred_work}
\section{Related work}


\subsection{Pattern Prediction over Event Streams}

\par Several approaches have been proposed to formalize the task of event patterns (complex events) prediction over time-evolving data streams.  One common way to formalize this task is to assume that the stream is a time-series of numerical values, and the goal is to predict at each time point $t$ the next observations at some future points $t+1$, $t+2$, etc., (or even the output of some function of future values)~\cite{montgomery_introduction_2015}. 


%The task of forecasting over time-evolving streams of data can be formulated in various ways and with varying assumptions.
%One common way to formalize this task is to assume that the stream is a time-series of numerical values, and the goal is to forecast at each time point $n$ the values at some future points $n+1$, $n+2$, etc., (or even the output of some function of future values). 
% This is the task of time-series forecasting ~\citet{montgomery_introduction_2015}.

\par Another way to formalize the prediction task is to view streams as sequences of events,
i.e., tuples of multiple attributes, such as \textit{id}, \textit{event type}, \textit{timestamp}, etc., and the goal is to predict future events or  patterns of events. In this work, we focus on the latter definition of forecasting (prediction of complex event patterns).  

\par A relevant work of the detection of event patterns task has been established in the field of temporal pattern mining. Where events are defined as 2-tuples of the form \((\mathit{EventType}, \mathit{Timestamp})\). The goal is to extract patterns of events in the form of association rules \cite{agrawal_mining_1993} or frequent episode rules \cite{mannila_discovery_1997}. 

\par These rule-based methods have been extended in order to be able to learn rules for predicting event patterns. For instance, in \cite{vilalta_predicting_2002}, an association rule mining technique is introduced. This technique works as following. Firstly, it extracts sets of event types that frequently lead to a target event (i.e., a rare event) within a time window of a fixed size. Then, it uses them to build a rule-based prediction model. Also,  ~\citet{weiss1998learning} proposed another rule-based method to predict rare events in a stream, using a genetic algorithm to find all predictive patterns to form prediction rules.  


\par On the other hand, ~\citet{laxman_stream_2008} have proposed a probabilistic model for calculating the probability of the immediately next event in the stream. Which is achieved by combining each frequent episode \cite{mannila1997discovery} that presents a partially-ordered set of event types, with a Hidden Markov Model (HMM). In addition, ~\citet{fahed_efficient_2014} have proposed episode rules mining algorithm which predicts distant events. By generating a set of episode rules in the form \(P \rightarrow Q \) where $P$ and $Q$ are two episodes. These rules have a minimal antecedent (in number of events) and temporally distant consequent.

\par In \cite{zhou_pattern_2015} a mining method is presented that finds the frequent sequential patterns in a stream of events, which are then used to generate prediction rules. The event stream is processed into batches, in each batch the events are consumed to find the prefix matches from the discovered frequent patterns, to predict future events using different strategies of prediction scoring.

% add more details about the cep
\par Event pattern prediction has also attracted some attention from the field of complex event processing (CEP), where the CEP system consumes a stream of low-level events to detect patterns of events (composite events) that are defined using pattern-based languages, these languages provide logic, sequence, and iteration operators such as SQl-like languages \cite{Cugola:2012:PFI:2187671.2187677}. 
\par One such early approach is presented in \cite{muthusamy_predictive_2010}, which is based on converting the complex event patterns to automata, and subsequently, Markov chains are used in order to estimate when a pattern is expected to be fully matched. 
\par Moreover, ~\citet{alevizos2017event} have recently presented a similar approach, where automata and Markov chains are employed in order to provide (future) time intervals during which a full match of the pattern is expected with a probability above a confidence threshold. In this thesis, we leverage this method as the base prediction model for each input event stream (see Section ~\ref{sec:Event-Forecasting-PMC}). 

\subsection{Distributed Online Learning}
\par In recent years, there have been many research efforts on the problem of distributed online learning \cite{langford2009slow,yan2013distributed,xiao2010dual,dekel2012optimal,kamp2014communication}.  A distributed online mini-batch prediction approach over multiple data streams has been proposed in \cite{dekel2012optimal}. This approach is based on a static synchronization method. The distributed learners/predictors periodically communicate  their local models with a central coordinator unit after consuming a fixed number of input samples/events (i.e., batch size $b$), in order to  create a global model parameters and share them between all learners. This work has been extended in \cite{kamp2014communication} by introducing a
dynamic synchronization scheme that reduces the required communication overhead. It can do so by making the local learners communicate their models only if they diverge from a reference model. In this work, we employ this protocol to synchronize the parameters of event patterns prediction models  over multiple event streams. 

\input{technological_background}


%\section{Related Work}
%
%In this section, we discuss the building blocks and techniques of our proposed approach. 
%
%
%\subsection{Event Forecasting with Pattern Markov Chains}
%
%Wayeb \cite{alevizos2017event} is a system that provides the ability to forecast the completion (i.e., full match) of defined patterns within an input stream of events, the proposed approach is based on the Pattern Markov Chain (PMC) framework \cite{nuel2008pattern} that provides a probabilistic model for the events pattern. The patterns are defined in  the form of regular expressions such as $p1=e1.*.(e3 | e4)$ ($p1$ is a pattern that starts with an event of type $e1$ followed by any event type, then ends with an event of type $e3$ or $e4$), which is transformed to deterministic finite automate $(DFA)$. The resulting $DFA$ is then used to construct a Markov Chain model, which  enables probabilistic inferences (i.e., probability of completion time interval) for the events pattern. Furthermore, the system implicitly is able to report the full matches of the foretasted pattern. We will employ it as the local prediction model in our system.  
%
%%discuss stream is stationary assumption
%\subsection{Distributed Online Learning}
%
%\par In \cite{kamp2014communication} a distributed prediction learning protocol over multiple data streams has been proposed. Their work focuses on providing a dynamic and communication efficient synchronization scheme of learning models between distributed learners, by triggering the synchronization process on local node only if its local model diverges from a global reference point. This dynamic scheme comes as an improvement of the static synchronization (i.e., mini-batch) scheme \cite{dekel2012optimal} that is based on making the local learners send their local models to a central site after processing a fixed number of input samples (i.e., batch). In both settings, a central node  called coordinator receives the local learning models, in order to construct an aggregated global model based on all local models and sends it back to all participating nodes.
%
%\par In our system we refer to the predicator/forecaster nodes as the distributed learners. Our integration plan is divided into two steps: first, employing the distributed online learning protocol with mini-batch synchronization scheme by combing the local models after certain batch size $b \in \N$. Second, reduce the number of times that local learners need to communicate their models by applying the dynamic synchronization protocol.


%\subsection{Clustering}
%
%\par Clustering a set of objects into partitioned groups is a popular unsupervised machine learning method, it aims to divide the objects into disjoint subsets of objects (i.e., clusters), where the objects that belong to the same cluster are similar to each other and different from the objects in other clusters, and the similarity between objects is typically measured by some distance function \cite{clustering}. One of the most common clustering algorithms $DBSCAN$ \cite{ester1996density}  which is density based clustering algorithm that identifies clusters with arbitrary shapes in large spatial databases. Likewise, $k$-$means$ \cite{k-means} is another common clustering algorithms, it divides $M$ points into $K$ clusters based on minimizing the sum of squares (i.e., distance) between the points within the same cluster.
%
%\par In this thesis, we are applying our approach on events streams of moving objects (i.e., vessels). Thus, we are focusing on the trajectory-based clustering methods. Fortunately, There are many trajectory-based clustering techniques that have been proposed in literature, we will have to examine and evaluate some of the proposed clustering techniques to be utilized in our system. 
%
% \par For instance, in \cite{liu2014knowledge} a density-based algorithm to cluster the moving trajectory points of vessels was introduced, they modified the definition of $Eps-neighborhood$ in the $DBSCAN$ algorithm to take into account the variance of the speed and direction. Moreover, Lee et al. \cite{lee2007trajectory} have proposed trajectories clustering algorithm (TRACLUS) that firstly splits the trajectory into  sequence of line segments,  which are then partitioned into clusters using s density-based clustering technique.    
%
%\par On the other hand, the authors of \cite{chen2005robust} have proposed a new distance function of trajectories of moving objects, which is called $Edit Distance on Real sequence (EDR)$ that measures the similarity between two trajectories of moving objects (e.g., $T_1$ and $T_2$ based on the cost is needed to change $T_1$ into $T_2$ by checking the matching of the element of the two trajectories.

%don't forget to mention concept drift.