% Prevent Line Breaking Inline Formula
\relpenalty=9999
\binoppenalty=9999

\usepackage[utf8]{inputenc}
\usepackage[english]{babel} % sets up english hyphenation
\usepackage{csquotes} % for language-dependent quotes in biblatex
\usepackage[unicode=true]{hyperref} % enables use of metadata for pdfs and hyperlinks within a document

\usepackage[numbers]{natbib}
\usepackage[usenames,dvipsnames,hyperref]{xcolor} % enables more advanced color support for hyperref
% links customization 
\hypersetup{colorlinks=true, %flag for prints
	hidelinks,  % this option would hide links for the print version of your thesis
	linkcolor=red!35!black,    %definition of the link color
	citecolor=green!35!black,  %definition of the cite color
	urlcolor=magenta!35!black, %definition of the url color
	%pdfauthor=, % Optional: Specify the author of the pdf
	%pdftitle=   % Optional: Specify the title within the pdf
}

\usepackage{algorithmicx}
\usepackage{algpseudocode}

\usepackage{verbatim} % for multiple line comment
\usepackage[final]{pdfpages} %include pdf files
\usepackage{amsmath}
\usepackage{amssymb}
\mathchardef\mhyphen="2D % Define a "math hyphen" 
\usepackage{amsthm}
\usepackage{eufrak}
\newtheorem*{definition*}{Definition}
\newtheorem{definition}{Definition}
\newtheorem{mylem}{Lemma}[section]
\newtheorem{mytheorem}{Theorem}[section]

\usepackage{subfiles} %This package is used for subfiles
\usepackage{tabu}     % provides advanced tables
\usepackage{array,multirow}

\usepackage{booktabs} % enables reference bookstyle tables
\usepackage[format=plain, labelfont=bf]{caption}
\usepackage[capitalize,noabbrev]{cleveref}
\usepackage{subcaption} % enables use multiple figures in a figure
\captionsetup{compatibility=false}
\usepackage{eurosym} %includes the euro symbol 
\usepackage{enumitem} % allows customization of enumeration and itemize environment
\usepackage{graphicx} % enables loading of graphics
\usepackage{tikz} % drawing vector graphics in latex
\usetikzlibrary{graphs,graphs.standard}
\usetikzlibrary{shapes.geometric,backgrounds}
\newcommand{\R}{\mathbb{R}}
\newcommand{\N}{\mathbb{N}}
\DeclareMathOperator*{\Exp}{\mathbb{E}}
\newcommand{\expAvgLoss}{\Exp\left[\overline{X}\right]}
\newcommand{\avgLoss}{\overline{X}}
\newcommand*\mean[1]{\bar{#1}}
\newcommand{\inputSpace}{\mathcal{Z}}
\newcommand{\dist}{\mathcal{D}}
\newcommand{\hilbertSpace}{\mathcal{H}}
\newcommand{\mapping}[3]{#1\!: #2 \to #3}
\newcommand{\mappingdef}[3]{#1\!\left (#2\right )=#3}
\newcommand{\prob}[1]{\mathbb{P}\!\left[#1\right]}
\newcommand{\regret}{R}
\newcommand{\hatv}[1]{\overset{\wedge}{\mathstrut#1}}

\newtheorem{thm}{Theorem}
\newtheorem{lem}[thm]{Lemma}
\newtheorem{prop}[thm]{Proposition}
\newtheorem{cor}[thm]{Corollary}
\newtheorem{exm}[thm]{Example}
\newtheorem{problem}[thm]{Problem}
\DeclareMathOperator*{\argmin}{\mathrm{argmin}}
 
\usepackage{setspace} % helps setup line spacing
%\onehalfspacing % increases linespacing to one and half
\usepackage{placeins} % provides FloatBarrier
%\usepackage[miktex]{gnuplottex} % gnuplot within latex. May be obsolete with pylab.
\usepackage[ruled,vlined,noend]{algorithm2e}
%algorithm package
%\linespread{1.1} % Definition of the linespread

\usepackage[tbtags]{mathtools}
\DeclareMathOperator*{\somefunc}{somefunc}
\SetKwInput{KwInput}{Input}
\SetKwInput{KwOutput}{Output}

%tikz helps to draw nice pictures with a lot of effort for advanced users
\usetikzlibrary{positioning, shapes, shadows, arrows, backgrounds}
\usepackage{verbatim}
\usepackage{tikz-3dplot}
\usepackage{forest}

\usepackage[export]{adjustbox}

%table of contents depth
\setcounter{secnumdepth}{4}
\setcounter{tocdepth}{4}% Allow only \chapter in ToC

%some definitions for the cref package
\crefname{algocf}{Algorithm}{Algorithms}
\crefname{table}{Table}{Tables}
\crefname{chapter}{Chapter}{Chapters}
\crefname{equation}{Equation}{Equations}
\crefname{section}{Section}{Sections}

\tikzset{
	tri/.style={
		draw,
		shape border rotate=90,
		isosceles triangle,
		isosceles triangle apex angle=60,
		node distance=1cm,
		minimum height=4em
	}
}

% paper

\def\dfar{$\mathit{DFA_{\mathcal{P}}}$}
\def\nfar{$\mathit{NFA_{\mathcal{P}}}$}
\def\dfasr{$\mathit{DFA_{\Sigma^{*}\cdot \mathcal{P}}}$}
\def\nfasr{$\mathit{NFA_{\Sigma^{*}\cdot \mathcal{P}}}$}
\def\pmcmr{$\mathit{PMC_{m}^{\mathcal{P}}\ }$}
\def\pmczeror{$\mathit{PMC_{\mathcal{P}}^{0}}$}
\def\pmconer{$\mathit{PMC_{\mathcal{P}}^{1}}$}
\def\pfc{$\mathit{P_{fc}}$}

\usepackage{listings}
\usepackage{color}

\definecolor{dkgreen}{rgb}{0,0.6,0}
\definecolor{gray}{rgb}{0.5,0.5,0.5}
\definecolor{mauve}{rgb}{0.58,0,0.82}

\lstset{
	language=Java,
	aboveskip=3mm,
	belowskip=3mm,
	showstringspaces=false,
	columns=flexible,
	basicstyle={\small\ttfamily},
	numbers=none,
	numberstyle=\tiny\color{gray},
	keywordstyle=\color{blue},
	commentstyle=\color{dkgreen},
	stringstyle=\color{mauve},
	breaklines=true,
	breakatwhitespace=true,
	tabsize=3,
	captionpos=b
}
% new command 

\linespread{1.095} % Definition of the linespread
% new
\usepackage{tikz}
\usetikzlibrary{arrows,automata}
\usepackage{filecontents}
\usepackage{natbib}
\usepackage{bibentry}
\nobibliography*